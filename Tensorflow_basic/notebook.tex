
% Default to the notebook output style

    


% Inherit from the specified cell style.




    
\documentclass[11pt]{article}

    
    
    \usepackage[T1]{fontenc}
    % Nicer default font (+ math font) than Computer Modern for most use cases
    \usepackage{mathpazo}

    % Basic figure setup, for now with no caption control since it's done
    % automatically by Pandoc (which extracts ![](path) syntax from Markdown).
    \usepackage{graphicx}
    % We will generate all images so they have a width \maxwidth. This means
    % that they will get their normal width if they fit onto the page, but
    % are scaled down if they would overflow the margins.
    \makeatletter
    \def\maxwidth{\ifdim\Gin@nat@width>\linewidth\linewidth
    \else\Gin@nat@width\fi}
    \makeatother
    \let\Oldincludegraphics\includegraphics
    % Set max figure width to be 80% of text width, for now hardcoded.
    \renewcommand{\includegraphics}[1]{\Oldincludegraphics[width=.8\maxwidth]{#1}}
    % Ensure that by default, figures have no caption (until we provide a
    % proper Figure object with a Caption API and a way to capture that
    % in the conversion process - todo).
    \usepackage{caption}
    \DeclareCaptionLabelFormat{nolabel}{}
    \captionsetup{labelformat=nolabel}

    \usepackage{adjustbox} % Used to constrain images to a maximum size 
    \usepackage{xcolor} % Allow colors to be defined
    \usepackage{enumerate} % Needed for markdown enumerations to work
    \usepackage{geometry} % Used to adjust the document margins
    \usepackage{amsmath} % Equations
    \usepackage{amssymb} % Equations
    \usepackage{textcomp} % defines textquotesingle
    % Hack from http://tex.stackexchange.com/a/47451/13684:
    \AtBeginDocument{%
        \def\PYZsq{\textquotesingle}% Upright quotes in Pygmentized code
    }
    \usepackage{upquote} % Upright quotes for verbatim code
    \usepackage{eurosym} % defines \euro
    \usepackage[mathletters]{ucs} % Extended unicode (utf-8) support
    \usepackage[utf8x]{inputenc} % Allow utf-8 characters in the tex document
    \usepackage{fancyvrb} % verbatim replacement that allows latex
    \usepackage{grffile} % extends the file name processing of package graphics 
                         % to support a larger range 
    % The hyperref package gives us a pdf with properly built
    % internal navigation ('pdf bookmarks' for the table of contents,
    % internal cross-reference links, web links for URLs, etc.)
    \usepackage{hyperref}
    \usepackage{longtable} % longtable support required by pandoc >1.10
    \usepackage{booktabs}  % table support for pandoc > 1.12.2
    \usepackage[inline]{enumitem} % IRkernel/repr support (it uses the enumerate* environment)
    \usepackage[normalem]{ulem} % ulem is needed to support strikethroughs (\sout)
                                % normalem makes italics be italics, not underlines
    

    
    
    % Colors for the hyperref package
    \definecolor{urlcolor}{rgb}{0,.145,.698}
    \definecolor{linkcolor}{rgb}{.71,0.21,0.01}
    \definecolor{citecolor}{rgb}{.12,.54,.11}

    % ANSI colors
    \definecolor{ansi-black}{HTML}{3E424D}
    \definecolor{ansi-black-intense}{HTML}{282C36}
    \definecolor{ansi-red}{HTML}{E75C58}
    \definecolor{ansi-red-intense}{HTML}{B22B31}
    \definecolor{ansi-green}{HTML}{00A250}
    \definecolor{ansi-green-intense}{HTML}{007427}
    \definecolor{ansi-yellow}{HTML}{DDB62B}
    \definecolor{ansi-yellow-intense}{HTML}{B27D12}
    \definecolor{ansi-blue}{HTML}{208FFB}
    \definecolor{ansi-blue-intense}{HTML}{0065CA}
    \definecolor{ansi-magenta}{HTML}{D160C4}
    \definecolor{ansi-magenta-intense}{HTML}{A03196}
    \definecolor{ansi-cyan}{HTML}{60C6C8}
    \definecolor{ansi-cyan-intense}{HTML}{258F8F}
    \definecolor{ansi-white}{HTML}{C5C1B4}
    \definecolor{ansi-white-intense}{HTML}{A1A6B2}

    % commands and environments needed by pandoc snippets
    % extracted from the output of `pandoc -s`
    \providecommand{\tightlist}{%
      \setlength{\itemsep}{0pt}\setlength{\parskip}{0pt}}
    \DefineVerbatimEnvironment{Highlighting}{Verbatim}{commandchars=\\\{\}}
    % Add ',fontsize=\small' for more characters per line
    \newenvironment{Shaded}{}{}
    \newcommand{\KeywordTok}[1]{\textcolor[rgb]{0.00,0.44,0.13}{\textbf{{#1}}}}
    \newcommand{\DataTypeTok}[1]{\textcolor[rgb]{0.56,0.13,0.00}{{#1}}}
    \newcommand{\DecValTok}[1]{\textcolor[rgb]{0.25,0.63,0.44}{{#1}}}
    \newcommand{\BaseNTok}[1]{\textcolor[rgb]{0.25,0.63,0.44}{{#1}}}
    \newcommand{\FloatTok}[1]{\textcolor[rgb]{0.25,0.63,0.44}{{#1}}}
    \newcommand{\CharTok}[1]{\textcolor[rgb]{0.25,0.44,0.63}{{#1}}}
    \newcommand{\StringTok}[1]{\textcolor[rgb]{0.25,0.44,0.63}{{#1}}}
    \newcommand{\CommentTok}[1]{\textcolor[rgb]{0.38,0.63,0.69}{\textit{{#1}}}}
    \newcommand{\OtherTok}[1]{\textcolor[rgb]{0.00,0.44,0.13}{{#1}}}
    \newcommand{\AlertTok}[1]{\textcolor[rgb]{1.00,0.00,0.00}{\textbf{{#1}}}}
    \newcommand{\FunctionTok}[1]{\textcolor[rgb]{0.02,0.16,0.49}{{#1}}}
    \newcommand{\RegionMarkerTok}[1]{{#1}}
    \newcommand{\ErrorTok}[1]{\textcolor[rgb]{1.00,0.00,0.00}{\textbf{{#1}}}}
    \newcommand{\NormalTok}[1]{{#1}}
    
    % Additional commands for more recent versions of Pandoc
    \newcommand{\ConstantTok}[1]{\textcolor[rgb]{0.53,0.00,0.00}{{#1}}}
    \newcommand{\SpecialCharTok}[1]{\textcolor[rgb]{0.25,0.44,0.63}{{#1}}}
    \newcommand{\VerbatimStringTok}[1]{\textcolor[rgb]{0.25,0.44,0.63}{{#1}}}
    \newcommand{\SpecialStringTok}[1]{\textcolor[rgb]{0.73,0.40,0.53}{{#1}}}
    \newcommand{\ImportTok}[1]{{#1}}
    \newcommand{\DocumentationTok}[1]{\textcolor[rgb]{0.73,0.13,0.13}{\textit{{#1}}}}
    \newcommand{\AnnotationTok}[1]{\textcolor[rgb]{0.38,0.63,0.69}{\textbf{\textit{{#1}}}}}
    \newcommand{\CommentVarTok}[1]{\textcolor[rgb]{0.38,0.63,0.69}{\textbf{\textit{{#1}}}}}
    \newcommand{\VariableTok}[1]{\textcolor[rgb]{0.10,0.09,0.49}{{#1}}}
    \newcommand{\ControlFlowTok}[1]{\textcolor[rgb]{0.00,0.44,0.13}{\textbf{{#1}}}}
    \newcommand{\OperatorTok}[1]{\textcolor[rgb]{0.40,0.40,0.40}{{#1}}}
    \newcommand{\BuiltInTok}[1]{{#1}}
    \newcommand{\ExtensionTok}[1]{{#1}}
    \newcommand{\PreprocessorTok}[1]{\textcolor[rgb]{0.74,0.48,0.00}{{#1}}}
    \newcommand{\AttributeTok}[1]{\textcolor[rgb]{0.49,0.56,0.16}{{#1}}}
    \newcommand{\InformationTok}[1]{\textcolor[rgb]{0.38,0.63,0.69}{\textbf{\textit{{#1}}}}}
    \newcommand{\WarningTok}[1]{\textcolor[rgb]{0.38,0.63,0.69}{\textbf{\textit{{#1}}}}}
    
    
    % Define a nice break command that doesn't care if a line doesn't already
    % exist.
    \def\br{\hspace*{\fill} \\* }
    % Math Jax compatability definitions
    \def\gt{>}
    \def\lt{<}
    % Document parameters
    \title{tf\_eagar(Define by run) ????}
    
    
    

    % Pygments definitions
    
\makeatletter
\def\PY@reset{\let\PY@it=\relax \let\PY@bf=\relax%
    \let\PY@ul=\relax \let\PY@tc=\relax%
    \let\PY@bc=\relax \let\PY@ff=\relax}
\def\PY@tok#1{\csname PY@tok@#1\endcsname}
\def\PY@toks#1+{\ifx\relax#1\empty\else%
    \PY@tok{#1}\expandafter\PY@toks\fi}
\def\PY@do#1{\PY@bc{\PY@tc{\PY@ul{%
    \PY@it{\PY@bf{\PY@ff{#1}}}}}}}
\def\PY#1#2{\PY@reset\PY@toks#1+\relax+\PY@do{#2}}

\expandafter\def\csname PY@tok@w\endcsname{\def\PY@tc##1{\textcolor[rgb]{0.73,0.73,0.73}{##1}}}
\expandafter\def\csname PY@tok@c\endcsname{\let\PY@it=\textit\def\PY@tc##1{\textcolor[rgb]{0.25,0.50,0.50}{##1}}}
\expandafter\def\csname PY@tok@cp\endcsname{\def\PY@tc##1{\textcolor[rgb]{0.74,0.48,0.00}{##1}}}
\expandafter\def\csname PY@tok@k\endcsname{\let\PY@bf=\textbf\def\PY@tc##1{\textcolor[rgb]{0.00,0.50,0.00}{##1}}}
\expandafter\def\csname PY@tok@kp\endcsname{\def\PY@tc##1{\textcolor[rgb]{0.00,0.50,0.00}{##1}}}
\expandafter\def\csname PY@tok@kt\endcsname{\def\PY@tc##1{\textcolor[rgb]{0.69,0.00,0.25}{##1}}}
\expandafter\def\csname PY@tok@o\endcsname{\def\PY@tc##1{\textcolor[rgb]{0.40,0.40,0.40}{##1}}}
\expandafter\def\csname PY@tok@ow\endcsname{\let\PY@bf=\textbf\def\PY@tc##1{\textcolor[rgb]{0.67,0.13,1.00}{##1}}}
\expandafter\def\csname PY@tok@nb\endcsname{\def\PY@tc##1{\textcolor[rgb]{0.00,0.50,0.00}{##1}}}
\expandafter\def\csname PY@tok@nf\endcsname{\def\PY@tc##1{\textcolor[rgb]{0.00,0.00,1.00}{##1}}}
\expandafter\def\csname PY@tok@nc\endcsname{\let\PY@bf=\textbf\def\PY@tc##1{\textcolor[rgb]{0.00,0.00,1.00}{##1}}}
\expandafter\def\csname PY@tok@nn\endcsname{\let\PY@bf=\textbf\def\PY@tc##1{\textcolor[rgb]{0.00,0.00,1.00}{##1}}}
\expandafter\def\csname PY@tok@ne\endcsname{\let\PY@bf=\textbf\def\PY@tc##1{\textcolor[rgb]{0.82,0.25,0.23}{##1}}}
\expandafter\def\csname PY@tok@nv\endcsname{\def\PY@tc##1{\textcolor[rgb]{0.10,0.09,0.49}{##1}}}
\expandafter\def\csname PY@tok@no\endcsname{\def\PY@tc##1{\textcolor[rgb]{0.53,0.00,0.00}{##1}}}
\expandafter\def\csname PY@tok@nl\endcsname{\def\PY@tc##1{\textcolor[rgb]{0.63,0.63,0.00}{##1}}}
\expandafter\def\csname PY@tok@ni\endcsname{\let\PY@bf=\textbf\def\PY@tc##1{\textcolor[rgb]{0.60,0.60,0.60}{##1}}}
\expandafter\def\csname PY@tok@na\endcsname{\def\PY@tc##1{\textcolor[rgb]{0.49,0.56,0.16}{##1}}}
\expandafter\def\csname PY@tok@nt\endcsname{\let\PY@bf=\textbf\def\PY@tc##1{\textcolor[rgb]{0.00,0.50,0.00}{##1}}}
\expandafter\def\csname PY@tok@nd\endcsname{\def\PY@tc##1{\textcolor[rgb]{0.67,0.13,1.00}{##1}}}
\expandafter\def\csname PY@tok@s\endcsname{\def\PY@tc##1{\textcolor[rgb]{0.73,0.13,0.13}{##1}}}
\expandafter\def\csname PY@tok@sd\endcsname{\let\PY@it=\textit\def\PY@tc##1{\textcolor[rgb]{0.73,0.13,0.13}{##1}}}
\expandafter\def\csname PY@tok@si\endcsname{\let\PY@bf=\textbf\def\PY@tc##1{\textcolor[rgb]{0.73,0.40,0.53}{##1}}}
\expandafter\def\csname PY@tok@se\endcsname{\let\PY@bf=\textbf\def\PY@tc##1{\textcolor[rgb]{0.73,0.40,0.13}{##1}}}
\expandafter\def\csname PY@tok@sr\endcsname{\def\PY@tc##1{\textcolor[rgb]{0.73,0.40,0.53}{##1}}}
\expandafter\def\csname PY@tok@ss\endcsname{\def\PY@tc##1{\textcolor[rgb]{0.10,0.09,0.49}{##1}}}
\expandafter\def\csname PY@tok@sx\endcsname{\def\PY@tc##1{\textcolor[rgb]{0.00,0.50,0.00}{##1}}}
\expandafter\def\csname PY@tok@m\endcsname{\def\PY@tc##1{\textcolor[rgb]{0.40,0.40,0.40}{##1}}}
\expandafter\def\csname PY@tok@gh\endcsname{\let\PY@bf=\textbf\def\PY@tc##1{\textcolor[rgb]{0.00,0.00,0.50}{##1}}}
\expandafter\def\csname PY@tok@gu\endcsname{\let\PY@bf=\textbf\def\PY@tc##1{\textcolor[rgb]{0.50,0.00,0.50}{##1}}}
\expandafter\def\csname PY@tok@gd\endcsname{\def\PY@tc##1{\textcolor[rgb]{0.63,0.00,0.00}{##1}}}
\expandafter\def\csname PY@tok@gi\endcsname{\def\PY@tc##1{\textcolor[rgb]{0.00,0.63,0.00}{##1}}}
\expandafter\def\csname PY@tok@gr\endcsname{\def\PY@tc##1{\textcolor[rgb]{1.00,0.00,0.00}{##1}}}
\expandafter\def\csname PY@tok@ge\endcsname{\let\PY@it=\textit}
\expandafter\def\csname PY@tok@gs\endcsname{\let\PY@bf=\textbf}
\expandafter\def\csname PY@tok@gp\endcsname{\let\PY@bf=\textbf\def\PY@tc##1{\textcolor[rgb]{0.00,0.00,0.50}{##1}}}
\expandafter\def\csname PY@tok@go\endcsname{\def\PY@tc##1{\textcolor[rgb]{0.53,0.53,0.53}{##1}}}
\expandafter\def\csname PY@tok@gt\endcsname{\def\PY@tc##1{\textcolor[rgb]{0.00,0.27,0.87}{##1}}}
\expandafter\def\csname PY@tok@err\endcsname{\def\PY@bc##1{\setlength{\fboxsep}{0pt}\fcolorbox[rgb]{1.00,0.00,0.00}{1,1,1}{\strut ##1}}}
\expandafter\def\csname PY@tok@kc\endcsname{\let\PY@bf=\textbf\def\PY@tc##1{\textcolor[rgb]{0.00,0.50,0.00}{##1}}}
\expandafter\def\csname PY@tok@kd\endcsname{\let\PY@bf=\textbf\def\PY@tc##1{\textcolor[rgb]{0.00,0.50,0.00}{##1}}}
\expandafter\def\csname PY@tok@kn\endcsname{\let\PY@bf=\textbf\def\PY@tc##1{\textcolor[rgb]{0.00,0.50,0.00}{##1}}}
\expandafter\def\csname PY@tok@kr\endcsname{\let\PY@bf=\textbf\def\PY@tc##1{\textcolor[rgb]{0.00,0.50,0.00}{##1}}}
\expandafter\def\csname PY@tok@bp\endcsname{\def\PY@tc##1{\textcolor[rgb]{0.00,0.50,0.00}{##1}}}
\expandafter\def\csname PY@tok@fm\endcsname{\def\PY@tc##1{\textcolor[rgb]{0.00,0.00,1.00}{##1}}}
\expandafter\def\csname PY@tok@vc\endcsname{\def\PY@tc##1{\textcolor[rgb]{0.10,0.09,0.49}{##1}}}
\expandafter\def\csname PY@tok@vg\endcsname{\def\PY@tc##1{\textcolor[rgb]{0.10,0.09,0.49}{##1}}}
\expandafter\def\csname PY@tok@vi\endcsname{\def\PY@tc##1{\textcolor[rgb]{0.10,0.09,0.49}{##1}}}
\expandafter\def\csname PY@tok@vm\endcsname{\def\PY@tc##1{\textcolor[rgb]{0.10,0.09,0.49}{##1}}}
\expandafter\def\csname PY@tok@sa\endcsname{\def\PY@tc##1{\textcolor[rgb]{0.73,0.13,0.13}{##1}}}
\expandafter\def\csname PY@tok@sb\endcsname{\def\PY@tc##1{\textcolor[rgb]{0.73,0.13,0.13}{##1}}}
\expandafter\def\csname PY@tok@sc\endcsname{\def\PY@tc##1{\textcolor[rgb]{0.73,0.13,0.13}{##1}}}
\expandafter\def\csname PY@tok@dl\endcsname{\def\PY@tc##1{\textcolor[rgb]{0.73,0.13,0.13}{##1}}}
\expandafter\def\csname PY@tok@s2\endcsname{\def\PY@tc##1{\textcolor[rgb]{0.73,0.13,0.13}{##1}}}
\expandafter\def\csname PY@tok@sh\endcsname{\def\PY@tc##1{\textcolor[rgb]{0.73,0.13,0.13}{##1}}}
\expandafter\def\csname PY@tok@s1\endcsname{\def\PY@tc##1{\textcolor[rgb]{0.73,0.13,0.13}{##1}}}
\expandafter\def\csname PY@tok@mb\endcsname{\def\PY@tc##1{\textcolor[rgb]{0.40,0.40,0.40}{##1}}}
\expandafter\def\csname PY@tok@mf\endcsname{\def\PY@tc##1{\textcolor[rgb]{0.40,0.40,0.40}{##1}}}
\expandafter\def\csname PY@tok@mh\endcsname{\def\PY@tc##1{\textcolor[rgb]{0.40,0.40,0.40}{##1}}}
\expandafter\def\csname PY@tok@mi\endcsname{\def\PY@tc##1{\textcolor[rgb]{0.40,0.40,0.40}{##1}}}
\expandafter\def\csname PY@tok@il\endcsname{\def\PY@tc##1{\textcolor[rgb]{0.40,0.40,0.40}{##1}}}
\expandafter\def\csname PY@tok@mo\endcsname{\def\PY@tc##1{\textcolor[rgb]{0.40,0.40,0.40}{##1}}}
\expandafter\def\csname PY@tok@ch\endcsname{\let\PY@it=\textit\def\PY@tc##1{\textcolor[rgb]{0.25,0.50,0.50}{##1}}}
\expandafter\def\csname PY@tok@cm\endcsname{\let\PY@it=\textit\def\PY@tc##1{\textcolor[rgb]{0.25,0.50,0.50}{##1}}}
\expandafter\def\csname PY@tok@cpf\endcsname{\let\PY@it=\textit\def\PY@tc##1{\textcolor[rgb]{0.25,0.50,0.50}{##1}}}
\expandafter\def\csname PY@tok@c1\endcsname{\let\PY@it=\textit\def\PY@tc##1{\textcolor[rgb]{0.25,0.50,0.50}{##1}}}
\expandafter\def\csname PY@tok@cs\endcsname{\let\PY@it=\textit\def\PY@tc##1{\textcolor[rgb]{0.25,0.50,0.50}{##1}}}

\def\PYZbs{\char`\\}
\def\PYZus{\char`\_}
\def\PYZob{\char`\{}
\def\PYZcb{\char`\}}
\def\PYZca{\char`\^}
\def\PYZam{\char`\&}
\def\PYZlt{\char`\<}
\def\PYZgt{\char`\>}
\def\PYZsh{\char`\#}
\def\PYZpc{\char`\%}
\def\PYZdl{\char`\$}
\def\PYZhy{\char`\-}
\def\PYZsq{\char`\'}
\def\PYZdq{\char`\"}
\def\PYZti{\char`\~}
% for compatibility with earlier versions
\def\PYZat{@}
\def\PYZlb{[}
\def\PYZrb{]}
\makeatother


    % Exact colors from NB
    \definecolor{incolor}{rgb}{0.0, 0.0, 0.5}
    \definecolor{outcolor}{rgb}{0.545, 0.0, 0.0}



    
    % Prevent overflowing lines due to hard-to-break entities
    \sloppy 
    % Setup hyperref package
    \hypersetup{
      breaklinks=true,  % so long urls are correctly broken across lines
      colorlinks=true,
      urlcolor=urlcolor,
      linkcolor=linkcolor,
      citecolor=citecolor,
      }
    % Slightly bigger margins than the latex defaults
    
    \geometry{verbose,tmargin=1in,bmargin=1in,lmargin=1in,rmargin=1in}
    
    

    \begin{document}
    
    
    \maketitle
    
    

    
    \hypertarget{tf-eager-tutorial-define-by-run}{%
\section{TF Eager Tutorial (Define by
Run)}\label{tf-eager-tutorial-define-by-run}}

정리 및 요약 by Ryah Shin

\href{https://research.googleblog.com/2017/10/eager-execution-imperative-define-by.html}{참고1:
구글 블로그}

\href{https://github.com/tensorflow/tensorflow/blob/master/tensorflow/contrib/eager/python/g3doc/guide.md}{참고2:
구글 깃허브}

\href{https://github.com/tensorflow/tensorflow\#installation}{Eager 모드
설치방법(TF Nightly)}

    \begin{Verbatim}[commandchars=\\\{\}]
{\color{incolor}In [{\color{incolor}2}]:} \PY{k+kn}{import} \PY{n+nn}{tensorflow} \PY{k}{as} \PY{n+nn}{tf}
        \PY{k+kn}{import} \PY{n+nn}{tensorflow}\PY{n+nn}{.}\PY{n+nn}{contrib}\PY{n+nn}{.}\PY{n+nn}{eager} \PY{k}{as} \PY{n+nn}{tfe}
        
        \PY{n}{tfe}\PY{o}{.}\PY{n}{enable\PYZus{}eager\PYZus{}execution}\PY{p}{(}\PY{p}{)}
\end{Verbatim}


    \hypertarget{gradients}{%
\subsection{Gradients}\label{gradients}}

    \begin{Verbatim}[commandchars=\\\{\}]
{\color{incolor}In [{\color{incolor}5}]:} \PY{k}{def} \PY{n+nf}{square}\PY{p}{(}\PY{n}{x}\PY{p}{)}\PY{p}{:}
            \PY{k}{return} \PY{n}{tf}\PY{o}{.}\PY{n}{multiply}\PY{p}{(}\PY{n}{x}\PY{p}{,} \PY{n}{x}\PY{p}{)}
        
        \PY{n}{grad} \PY{o}{=} \PY{n}{tfe}\PY{o}{.}\PY{n}{gradients\PYZus{}function}\PY{p}{(}\PY{n}{square}\PY{p}{)}
        
        \PY{n+nb}{print}\PY{p}{(}\PY{n}{square}\PY{p}{(}\PY{l+m+mf}{3.}\PY{p}{)}\PY{p}{)}
        \PY{n+nb}{print}\PY{p}{(}\PY{n}{grad}\PY{p}{(}\PY{l+m+mf}{3.}\PY{p}{)}\PY{p}{)} \PY{c+c1}{\PYZsh{}x\PYZca{}2 \PYZhy{}\PYZgt{} 2x \PYZhy{}\PYZgt{} 6}
        
        \PY{c+c1}{\PYZsh{}2차 gradients\PYZus{}function}
        \PY{n}{gradgrad} \PY{o}{=} \PY{n}{tfe}\PY{o}{.}\PY{n}{gradients\PYZus{}function}\PY{p}{(}\PY{k}{lambda} \PY{n}{x}\PY{p}{:} \PY{n}{grad}\PY{p}{(}\PY{n}{x}\PY{p}{)}\PY{p}{[}\PY{l+m+mi}{0}\PY{p}{]}\PY{p}{)}
        
        \PY{n+nb}{print}\PY{p}{(}\PY{n}{gradgrad}\PY{p}{(}\PY{l+m+mf}{3.}\PY{p}{)}\PY{p}{)}
        
        \PY{k}{def} \PY{n+nf}{abs}\PY{p}{(}\PY{n}{x}\PY{p}{)}\PY{p}{:}
            \PY{k}{return} \PY{n}{x} \PY{k}{if} \PY{n}{x} \PY{o}{\PYZgt{}} \PY{l+m+mf}{0.} \PY{k}{else} \PY{o}{\PYZhy{}}\PY{n}{x}
        
        \PY{n}{grad} \PY{o}{=} \PY{n}{tfe}\PY{o}{.}\PY{n}{gradients\PYZus{}function}\PY{p}{(}\PY{n+nb}{abs}\PY{p}{)}
        
        \PY{n+nb}{print}\PY{p}{(}\PY{n}{grad}\PY{p}{(}\PY{l+m+mf}{2.0}\PY{p}{)}\PY{p}{)}  \PY{c+c1}{\PYZsh{} [1.]}
        \PY{n+nb}{print}\PY{p}{(}\PY{n}{grad}\PY{p}{(}\PY{o}{\PYZhy{}}\PY{l+m+mf}{2.0}\PY{p}{)}\PY{p}{)} \PY{c+c1}{\PYZsh{} [\PYZhy{}1.]}
\end{Verbatim}


    \begin{Verbatim}[commandchars=\\\{\}]
tf.Tensor(9.0, shape=(), dtype=float32)
[<tf.Tensor: id=59, shape=(), dtype=float32, numpy=6.0>]
[<tf.Tensor: id=75, shape=(), dtype=float32, numpy=2.0>]
[<tf.Tensor: id=84, shape=(), dtype=float32, numpy=1.0>]
[<tf.Tensor: id=94, shape=(), dtype=float32, numpy=-1.0>]

    \end{Verbatim}

    \hypertarget{custom-grad}{%
\subsection{Custom Grad}\label{custom-grad}}

Custom Gradient 제작하기.

주로 cross entropy나 log likelyhood에 쓰이는 예제로 log(1 + e\^{}x) 제작

    \begin{Verbatim}[commandchars=\\\{\}]
{\color{incolor}In [{\color{incolor}10}]:} \PY{k}{def} \PY{n+nf}{log1pexp}\PY{p}{(}\PY{n}{x}\PY{p}{)}\PY{p}{:}
             \PY{k}{return} \PY{n}{tf}\PY{o}{.}\PY{n}{log}\PY{p}{(}\PY{l+m+mi}{1} \PY{o}{+} \PY{n}{tf}\PY{o}{.}\PY{n}{exp}\PY{p}{(}\PY{n}{x}\PY{p}{)}\PY{p}{)}
         
         \PY{n}{grad\PYZus{}log1pexp} \PY{o}{=} \PY{n}{tfe}\PY{o}{.}\PY{n}{gradients\PYZus{}function}\PY{p}{(}\PY{n}{log1pexp}\PY{p}{)}
         
         \PY{n+nb}{print}\PY{p}{(}\PY{n}{grad\PYZus{}log1pexp}\PY{p}{(}\PY{l+m+mf}{0.}\PY{p}{)}\PY{p}{)} \PY{c+c1}{\PYZsh{} [0.5]}
         
         \PY{n+nb}{print}\PY{p}{(}\PY{n}{grad\PYZus{}log1pexp}\PY{p}{(}\PY{l+m+mf}{100.}\PY{p}{)}\PY{p}{)} \PY{c+c1}{\PYZsh{} x = 100, nan}
\end{Verbatim}


    \begin{Verbatim}[commandchars=\\\{\}]
[<tf.Tensor: id=115, shape=(), dtype=float32, numpy=0.5>]
[<tf.Tensor: id=126, shape=(), dtype=float32, numpy=nan>]

    \end{Verbatim}

    \begin{Verbatim}[commandchars=\\\{\}]
{\color{incolor}In [{\color{incolor}13}]:} \PY{n+nd}{@tfe}\PY{o}{.}\PY{n}{custom\PYZus{}gradient}
         \PY{k}{def} \PY{n+nf}{log1pexp}\PY{p}{(}\PY{n}{x}\PY{p}{)}\PY{p}{:}
             \PY{n}{e} \PY{o}{=} \PY{n}{tf}\PY{o}{.}\PY{n}{exp}\PY{p}{(}\PY{n}{x}\PY{p}{)}
             \PY{k}{def} \PY{n+nf}{grad}\PY{p}{(}\PY{n}{dy}\PY{p}{)}\PY{p}{:}
                 \PY{k}{return} \PY{n}{dy} \PY{o}{*} \PY{p}{(}\PY{l+m+mi}{1} \PY{o}{\PYZhy{}} \PY{l+m+mi}{1} \PY{o}{/} \PY{p}{(}\PY{l+m+mi}{1} \PY{o}{+} \PY{n}{e}\PY{p}{)}\PY{p}{)}
             \PY{k}{return} \PY{n}{tf}\PY{o}{.}\PY{n}{log}\PY{p}{(}\PY{l+m+mi}{1} \PY{o}{+} \PY{n}{e}\PY{p}{)}\PY{p}{,} \PY{n}{grad}
         \PY{n}{grad\PYZus{}log1pexp} \PY{o}{=} \PY{n}{tfe}\PY{o}{.}\PY{n}{gradients\PYZus{}function}\PY{p}{(}\PY{n}{log1pexp}\PY{p}{)}
         
         \PY{c+c1}{\PYZsh{} Gradient at x = 0 works as before.}
         \PY{n+nb}{print}\PY{p}{(}\PY{n}{grad\PYZus{}log1pexp}\PY{p}{(}\PY{l+m+mf}{0.}\PY{p}{)}\PY{p}{)}
         \PY{c+c1}{\PYZsh{} [0.5]}
         \PY{c+c1}{\PYZsh{} And now gradient computation at x=100 works as well.}
         \PY{n+nb}{print}\PY{p}{(}\PY{n}{grad\PYZus{}log1pexp}\PY{p}{(}\PY{l+m+mf}{100.}\PY{p}{)}\PY{p}{)}
         \PY{c+c1}{\PYZsh{} [1.0]}
\end{Verbatim}


    \begin{Verbatim}[commandchars=\\\{\}]
[<tf.Tensor: id=138, shape=(), dtype=float32, numpy=0.5>]
[<tf.Tensor: id=150, shape=(), dtype=float32, numpy=1.0>]

    \end{Verbatim}

    \hypertarget{building-models}{%
\section{Building Models}\label{building-models}}

MNIST 2 Layer모델을 간단하게 Class로 만드는 예제

tfe.Network: 기본적으로 layer의 Container역할을 하여, 다른 NW객체에
임비디드 되어 NW객체가 된다.

추가로, inspection, saving, \& restoring에 도움을 준다.

    \begin{Verbatim}[commandchars=\\\{\}]
{\color{incolor}In [{\color{incolor}4}]:} \PY{k}{class} \PY{n+nc}{MNISTModel}\PY{p}{(}\PY{n}{tfe}\PY{o}{.}\PY{n}{Network}\PY{p}{)}\PY{p}{:}
            \PY{k}{def} \PY{n+nf}{\PYZus{}\PYZus{}init\PYZus{}\PYZus{}}\PY{p}{(}\PY{n+nb+bp}{self}\PY{p}{)}\PY{p}{:}
                \PY{n+nb}{super}\PY{p}{(}\PY{n}{MNISTModel}\PY{p}{,} \PY{n+nb+bp}{self}\PY{p}{)}\PY{o}{.}\PY{n+nf+fm}{\PYZus{}\PYZus{}init\PYZus{}\PYZus{}}\PY{p}{(}\PY{p}{)}
                \PY{n+nb+bp}{self}\PY{o}{.}\PY{n}{layer1} \PY{o}{=} \PY{n+nb+bp}{self}\PY{o}{.}\PY{n}{track\PYZus{}layer}\PY{p}{(}\PY{n}{tf}\PY{o}{.}\PY{n}{layers}\PY{o}{.}\PY{n}{Dense}\PY{p}{(}\PY{n}{units}\PY{o}{=}\PY{l+m+mi}{10}\PY{p}{)}\PY{p}{)}
                \PY{n+nb+bp}{self}\PY{o}{.}\PY{n}{layer2} \PY{o}{=} \PY{n+nb+bp}{self}\PY{o}{.}\PY{n}{track\PYZus{}layer}\PY{p}{(}\PY{n}{tf}\PY{o}{.}\PY{n}{layers}\PY{o}{.}\PY{n}{Dense}\PY{p}{(}\PY{n}{units}\PY{o}{=}\PY{l+m+mi}{10}\PY{p}{)}\PY{p}{)}
            \PY{k}{def} \PY{n+nf}{call}\PY{p}{(}\PY{n+nb+bp}{self}\PY{p}{,} \PY{n+nb}{input}\PY{p}{)}\PY{p}{:}
                \PY{l+s+sd}{\PYZdq{}\PYZdq{}\PYZdq{}모델 실행\PYZdq{}\PYZdq{}\PYZdq{}}
                \PY{n}{result} \PY{o}{=} \PY{n+nb+bp}{self}\PY{o}{.}\PY{n}{layer1}\PY{p}{(}\PY{n+nb}{input}\PY{p}{)}
                \PY{n}{result} \PY{o}{=} \PY{n+nb+bp}{self}\PY{o}{.}\PY{n}{layer2}\PY{p}{(}\PY{n}{result}\PY{p}{)}
                \PY{k}{return} \PY{n}{result}
            
        \PY{c+c1}{\PYZsh{}placeholder나 session에 대한 기능이 없고, input을 pass되면 자동으로 세팅 됨}
\end{Verbatim}


    \begin{Verbatim}[commandchars=\\\{\}]
{\color{incolor}In [{\color{incolor}5}]:} \PY{c+c1}{\PYZsh{} 테스트 데이터셋 생성하기}
        \PY{n}{model} \PY{o}{=} \PY{n}{MNISTModel}\PY{p}{(}\PY{p}{)}
        \PY{n}{batch} \PY{o}{=} \PY{n}{tf}\PY{o}{.}\PY{n}{zeros}\PY{p}{(}\PY{p}{[}\PY{l+m+mi}{1}\PY{p}{,} \PY{l+m+mi}{1}\PY{p}{,} \PY{l+m+mi}{784}\PY{p}{]}\PY{p}{)}
        \PY{n+nb}{print}\PY{p}{(}\PY{n}{batch}\PY{o}{.}\PY{n}{shape}\PY{p}{)}
        \PY{n}{result} \PY{o}{=} \PY{n}{model}\PY{p}{(}\PY{n}{batch}\PY{p}{)}
        \PY{n+nb}{print}\PY{p}{(}\PY{n}{result}\PY{p}{)}
\end{Verbatim}


    \begin{Verbatim}[commandchars=\\\{\}]
(1, 1, 784)
tf.Tensor([[[ 0.  0.  0.  0.  0.  0.  0.  0.  0.  0.]]], shape=(1, 1, 10), dtype=float32)

    \end{Verbatim}

    \begin{Verbatim}[commandchars=\\\{\}]
{\color{incolor}In [{\color{incolor}8}]:} \PY{c+c1}{\PYZsh{}학습을 위한 loss func, grad, 그리고 업데이트}
        
        \PY{c+c1}{\PYZsh{}1. loss func}
        \PY{k}{def} \PY{n+nf}{loss\PYZus{}function}\PY{p}{(}\PY{n}{model}\PY{p}{,} \PY{n}{x}\PY{p}{,} \PY{n}{y}\PY{p}{)}\PY{p}{:}
            \PY{n}{y\PYZus{}} \PY{o}{=} \PY{n}{model}\PY{p}{(}\PY{n}{x}\PY{p}{)}
            \PY{k}{return} \PY{n}{tf}\PY{o}{.}\PY{n}{nn}\PY{o}{.}\PY{n}{softmax\PYZus{}cross\PYZus{}entropy\PYZus{}with\PYZus{}logits}\PY{p}{(}\PY{n}{labels}\PY{o}{=}\PY{n}{y}\PY{p}{,} \PY{n}{logits}\PY{o}{=}\PY{n}{y\PYZus{}}\PY{p}{)}
        
        \PY{c+c1}{\PYZsh{}2. training loop}
        \PY{c+c1}{\PYZsh{}implicit\PYZus{}gradients(): 모든 TF 값에 대한 미분을 계산한다.}
        
        \PY{n}{optimizer} \PY{o}{=} \PY{n}{tf}\PY{o}{.}\PY{n}{train}\PY{o}{.}\PY{n}{GradientDescentOptimizer}\PY{p}{(}\PY{n}{learning\PYZus{}rate} \PY{o}{=}\PY{l+m+mf}{0.001}\PY{p}{)}
        \PY{k}{for} \PY{p}{(}\PY{n}{x}\PY{p}{,} \PY{n}{y}\PY{p}{)} \PY{o+ow}{in} \PY{n}{tfe}\PY{o}{.}\PY{n}{Iterator}\PY{p}{(}\PY{n}{dataset}\PY{p}{)}\PY{p}{:}
            \PY{n}{grads} \PY{o}{=} \PY{n}{tfe}\PY{o}{.}\PY{n}{implicit\PYZus{}gradients}\PY{p}{(}\PY{n}{loss\PYZus{}function}\PY{p}{)}\PY{p}{(}\PY{n}{model}\PY{p}{,} \PY{n}{x}\PY{p}{,} \PY{n}{y}\PY{p}{)}
            \PY{n}{optimizer}\PY{o}{.}\PY{n}{apply\PYZus{}gradients}\PY{p}{(}\PY{n}{grads}\PY{p}{)}
\end{Verbatim}


    \begin{Verbatim}[commandchars=\\\{\}]

        ---------------------------------------------------------------------------

        AttributeError                            Traceback (most recent call last)

        <ipython-input-8-5afbcb6c178e> in <module>()
         10 
         11 optimizer = tf.train.GradientDescentOptimizer(learning\_rate =0.001)
    ---> 12 for (x, y) in tfe.Iterator(batch):
         13     grads = tfe.implicit\_gradients(loss\_function)(model, x, y)
         14     optimizer.apply\_gradients(grads)


        \textasciitilde{}/tf\_nightly/lib/python3.6/site-packages/tensorflow/contrib/eager/python/datasets.py in \_\_init\_\_(self, dataset)
         75           format(type(self)))
         76     with ops.device("/device:CPU:0"):
    ---> 77       ds\_variant = dataset.\_as\_variant\_tensor()  \# pylint: disable=protected-access
         78       self.\_output\_types = dataset.output\_types
         79       self.\_output\_shapes = dataset.output\_shapes


        AttributeError: 'EagerTensor' object has no attribute '\_as\_variant\_tensor'

    \end{Verbatim}

    \begin{Verbatim}[commandchars=\\\{\}]
{\color{incolor}In [{\color{incolor} }]:} \PY{c+c1}{\PYZsh{}GPU 사용법}
        \PY{c+c1}{\PYZsh{}optimizer.min을 통해서, 짧게 작성하였지만, apply\PYZus{}gradients()기능을 써도 가능}
        
        \PY{k}{with} \PY{n}{tf}\PY{o}{.}\PY{n}{device}\PY{p}{(}\PY{l+s+s2}{\PYZdq{}}\PY{l+s+s2}{/gpu:0}\PY{l+s+s2}{\PYZdq{}}\PY{p}{)}\PY{p}{:}
            \PY{k}{for} \PY{p}{(}\PY{n}{x}\PY{p}{,} \PY{n}{y}\PY{p}{)} \PY{o+ow}{in} \PY{n}{tfe}\PY{o}{.}\PY{n}{Iterator}\PY{p}{(}\PY{n}{dataset}\PY{p}{)}\PY{p}{:}
                \PY{n}{optimizer}\PY{o}{.}\PY{n}{minimize}\PY{p}{(}\PY{k}{lambda}\PY{p}{:} \PY{n}{loss\PYZus{}function}\PY{p}{(}\PY{n}{model}\PY{p}{,} \PY{n}{x}\PY{p}{,} \PY{n}{y}\PY{p}{)}\PY{p}{)}
\end{Verbatim}


    \hypertarget{using-ear-with-graphs}{%
\subsection{Using Ear with Graphs}\label{using-ear-with-graphs}}

\begin{itemize}
\tightlist
\item
  Eagar 자체는 개발하고 디버깅 할 때 좋은 기능을 가지지만, Tensorflow
  graph형식이 분산 학습, 성능 최적화, 상용개발에 더 적합
\item
  현재 모델을 graph형태로 변경하기 위해서는, eager를 disable하고
  실행하면 됨
\item
  관련 예제 코드:
  \href{https://github.com/tensorflow/tensorflow/tree/master/tensorflow/contrib/eager/python/examples/mnist}{MNIST
  with Eager}
\item
  위의 예제 코드는 checkpoints를 저장하고 불러올 수 있기 때문에,
  상용에도 적합하다.
\end{itemize}

\hypertarget{uxd604uxc7ac-uxd65cuxc6a9uxd558uxace0-uxc788uxb294-uxcf54uxb4dcuxc758-uxbcc0uxd654}{%
\subsection{현재 활용하고 있는 코드의
변화}\label{uxd604uxc7ac-uxd65cuxc6a9uxd558uxace0-uxc788uxb294-uxcf54uxb4dcuxc758-uxbcc0uxd654}}

\begin{itemize}
\tightlist
\item
  현재 사용하고 있는 데이터에서, 형태를 tf.data로 변경하는것을 추천
  드립니다.

  \begin{itemize}
  \tightlist
  \item
    \href{https://developers.googleblog.com/2017/09/introducing-tensorflow-datasets.html}{참고링크
    1}
  \item
    \href{https://www.tensorflow.org/programmers_guide/datasets}{참고링크
    2}
  \end{itemize}
\item
  tf.layer.Conv2D()와 같은 Object-oriented 기능 활용을 추천 (Explict
  storage for variables
\item
  대부분의 모델이 eagar로 활용이 가능하지만, dynamic 모델에 대한 control
  flow같은 경우는 추가로 검토가 필요
\item
  tfe.enable\_eagar\_execution()을 활용하면, 끌수가 없기 때문에 Python
  세션을 재시작 추천
\end{itemize}


    % Add a bibliography block to the postdoc
    
    
    
    \end{document}
